% Clase
\documentclass[12pt,a4paper,spanish,oneside]{report}

% Órdenes auxiliares
\input{inc/includes.tex}

\usepackage{anysize}
\usepackage{amsthm}
\usepackage{verbatim}
\usepackage{multirow}
\usepackage{setspace}
\usepackage{tocbibind}
\usepackage{times}

%Para las secciones y las subseciones
\makeatletter
\renewcommand{\thesection}{\@arabic\c@section}
\renewcommand{\section}{
  \@startsection{section}{1}{0mm}{\baselineskip}
  {2.5mm}{\huge\bf}
}
\renewcommand{\subsection}{
  \@startsection{subsection}{2}{0mm}{2mm}
  {4.0mm}{\Large\bf}
}
\renewcommand{\subsubsection}{
  \@startsection{subsubsection}{3}{0mm}{2mm}
  {0.1mm}{\large\bf\emph}
}
\makeatother

\theoremstyle{plain} \newtheorem{nota}{Nota}

%color de las url
\hypersetup{urlcolor=blue}
% Encabezado y pie de página
\encabezado

\begin{document}
% Portada
\marginsize{2.5cm}{2cm}{3cm}{2cm}
\portada{Ingeniería en Informática}
{\emph{FrameBook}}{Red social familiar para personas dependientes}{Anteproyecto Fin de Carrera}{Autor: Juan Miguel Torres Triviño\\Director: Ramón Hervás Lucas}{Febrero, 2012}
\marginsize{3.5cm}{2cm}{2.5cm}{2.5cm}
% Licencia
\licencia{Juan Miguel Torres Triviño} 
% Índices
\renewcommand*{\contentsname}{Índice de Contenidos}
\tableofcontents
%\listoftables
%\listoffigures
\newpage
\section{Objetivos}
\subsection{Introducción}
\begin{spacing}{1.5}
El objetivo primordial de la Ambient Assisting Living (AAL \cite{AAL}) es 
facilitar y mejorar la calidad de vida en personas mayores o dependientes de 
otras mediante el uso de tecnologías de información y comunicación. Dentro del
ámbito de la AAL surgen multitud de aplicaciones móviles que se centran en 
distintos aspectos de la vida como las relaciones sociales. Este tipo de 
relaciones han tomado un rumbo diferente desde que entrasen en escena las redes
 sociales en la Web, como por ejemplo Twitter o Facebook entre otras. 

Gracias a la integración de nuevas tecnologías en los dispositivos móviles como
códigos QR \cite{QR}, cámara, acelerómetro y conexión a Internet se han 
desarrollado aplicaciones que nos aportan información de redes sociales y 
permiten interactuar con ellas.
\end{spacing}
\subsection{Objetivo general}
\begin{spacing}{1.5}
El objetivo general del proyecto es el desarrollo de una aplicación en la 
plataforma Android que permita a las personas dependientes interactuar con la 
red social Facebook. La aplicación utiliza interfaces sencillas y cómodas para
esa interacción.
\end{spacing}
\subsection{Objetivos específicos}
\begin{spacing}{1.5}
Se plantean una serie de objetivos específicos en los que se centra el trabajo a
lo largo del proyecto:
\begin{itemize}
\item Desarrollo de una aplicación móvil en la plataforma Android que permita en
todo momento acceder a la información personal de la red social Facebook. Se 
realiza a través de servicios web que permiten en cualquier instante acceder a 
esos datos.
\item Realización de una interfaz invisible y sencilla para la aplicación móvil.
La interacción es llevada a cabo por el uso de códigos QR y metáforas
 que representan objetos cotidianos y su correspondencia con los posibles 
elementos de la red social. Además la representación de los datos 
proporcionados es intuitiva y fácil de entender para el usuario.
\item Uso de librerías que simplifiquen algunos aspectos de la comunicación con 
el dispositivo móvil.
\end{itemize}
\end{spacing}
\subsection{Antecedentes y motivación}
\begin{spacing}{1.5}
Actualmente existe una gran variedad de aplicaciones sobre redes sociales para 
dispositivos móviles, aunque casi ninguna enfocada al ámbito de las personas
dependientes. \emph{Eldercare} \cite{CTW} surge como aplicación móvil para el 
seguimiento de distintas actividades sobre la salud en personas mayores. Como 
se puede ver en la figura \ref{ct} se muestran las distintas acciones que se 
pueden realizar sobre la aplicación. 
\imagen{img/caretwitter.png}{7}{Aplicación CareTwitter.}{ct}
Utiliza la tecnología NFC \footnote[1]{\emph{NFC:} Near Field Communication.} 
para la comunicación con el móvil y posteriormente muestra esos resultados en 
la red social Twitter, como se puede ver en la figura \ref{tw}.
\imagen{img/twitter.png}{7}{Twitter.}{tw}
Las redes sociales continúan avanzando en Internet a pasos agigantados, 
especialmente dentro de lo que se ha denominado Web 2.0 y Web 3.0, de entre 
todas la que más ha aumentado en los últimos años es Facebook. Según el ranking
\emph{Alexa}, generado por la empresa Alexa Interntet \cite{ALX}, la red social
 Facebook es el segundo sitio más visitado en Internet y tiene unos 800.000 
millones de usuarios en 2012. Estas estadísticas informan de la importancia de
Facebook en los usuarios de Internet y por tanto, la necesidad de acercar esta
experiencia a personas mayores o dependientes.
\end{spacing}
\section{Métodos y fases de trabajo}
\subsection{Metodología de desarrollo}
\begin{spacing}{1.5}
Para el desarrollo del proyecto se utiliza el Proceso Unificado de Desarrollo
\cite{PUD}. Consiste en un proceso iterativo e incremental para dividir el 
trabajo en partes más concretas y sencillas de realizar. Se caracteriza por 
estar dirigido hacia casos de uso y estar centrado en la arquitectura. El 
desarrollo se divide en una serie de fases, que a su vez son divididas en 
iteraciones, esto tiene como objetivo la planificación de diferentes 
actividades para cada iteración. Al final de cada iteración se producen varios 
entregables de tipo documentación o software que engloban uno o varios casos de
 uso.
\end{spacing}
\subsection{Fases del desarrollo}
\begin{spacing}{1.5}
Las fases para el desarrollo completo de la aplicación, ordenadas 
cronológicamente, son las siguientes:
\begin{description}
\item [Estado del arte.] En esta fase se realiza un análisis de la situación 
actual de la temática del proyecto. Se tienen en cuenta tanto investigaciones 
académicas como aplicaciones comerciales.
\item [Desarrollo software.] En esta fase se procede a la construcción de la 
aplicación siguiendo una serie de iteraciones, estas dependientes del número de
casos de uso. Cada iteración consta de las siguientes fases:
\begin{itemize}
\item \emph{Recopilación de requisitos.} En esta fase se concretan los 
requisitos que debe tener la aplicación a desarrollar.
\item \emph{Análisis.} Tras la recogida de requisitos en la fase anterior, se
realiza un análisis de los diferentes casos de uso de los que constará el 
sistema final. Se utiliza el lenguaje unificado de modelado (UML) para 
representar dichos casos de uso.
\item \emph{Diseño.} En esta fase se profundiza en cada una de los casos de uso
del sistema, diseñando sus respectivos componentes. También se utiliza el 
lenguaje unificado de modelado (UML) para representar dichos diseños.
\item \emph{Implementación.} En esta fase se realiza la implementación de los 
casos de uso, basándose en los diseños de la fase anterior. Se utilizan los 
lenguajes de programación y librerías más idóneos para esta fase, que 
previamentes han sido elegidos.
\item \emph{Pruebas.} Se prueba la aplicación en diferentes dispositivos que 
han sido escogidos con anterioridad. Las pruebas se realizan de la forma más 
exigente posible, es decir, en cada iteración se llevan a cabo pruebas 
unitarias, de regresión y de integración.
\end{itemize}
\item[Documentación.] Según se va desarrollando la aplicación se va redactando
un documentación que describe detalladamente cada una de las fases mencionadas
anteriormente. Incluye un manual de usuario y los diagramas de diseño 
realizados.
\end{description}
\end{spacing}
\section{Medios a utilizar}
\subsection{Medios hardware}
\begin{spacing}{1.5}
Se utilizan varios dispositivos móviles con la plataforma Android entre ellos el
Geeks\-phone One y el Galaxy Tab 7'.
\end{spacing}
\subsection{Medios software}
\begin{spacing}{1.5}
Se utiliza la plataforma Android \cite{AND}para el desarrollo de la aplicación
 móvil. El lenguaje de programación elegido es Java \cite{JAVA}. Además se hace
 uso de la API \footnote[2]{\emph{API:} Application Programming Interface.} de 
Facebook \cite{FB} para poder acceder a sus bases de datos y así poder 
modificar sus vistas que se muestran al usuario final.

El entorno integrado de desarrollo que se utiliza es Eclipse\cite{EC} que 
permite integrar el plugin ADT (Android Development Tools) para el desarrollo 
en la plataforma Android. En el caso de almacenar datos en el móvil se hace uso
 tanto de SQLite \cite{SQL} como de archivos XML \cite{XML}.
\end{spacing}
\subsection{Otros medios}
\begin{spacing}{1.5}
Se utiliza el lenguaje unificado de modelado (UML) \cite{UML} para la 
realización de los diagramas de diseño y análisis. Para el control de versiones
se utiliza el sistema distribuido Git \cite{GIT}y almacenado en Github 
\cite{GITH}, además para la gestión del proyecto se utiliza Microsoft Project 
\cite{MS}  y para la documentación escrita el lenguaje de composición de textos 
\LaTeX{}. Además se utilizan las librerías que se consideren necesarias para
mejorar la interacción con el dispositivo móvil.
\end{spacing}
\begin{thebibliography}{99}
\bibitem{AAL} Ambient Assisted Living Joint Programme. 
Web: \url{http://www.aal-europe.eu}
\bibitem{QR} Sitio oficial códigos QR. Web: \url{http://www.qrplanet.com}
\bibitem{CTW} Diego López-de-Ipiña, Ignacio Díaz-de-Sarralde, Javier 
García-Zubia. \emph{An Ambient Assisted Living Platform Integrating RFID Data-
on-Tag Care Annotations and Twitter.} Journal of Universal Computer Science, 
vol. 16, no. 12 (2010).
\bibitem{ALX} Sitio oficial de Alexa Internet. Web: \url{http://www.alexa.com}
\bibitem{PUD} Ivar Jacobson, Grady Booch, y James Rumbaugh. \emph{The Unified 
Software Development Process}. Addison-Wesley Professional, 1999.
\bibitem{UML} Sitio oficial de UML. Web: \url{http://www.uml.org}
\bibitem{AND} Sitio oficial de Android. Web: \url{http://www.android.com}
\bibitem{JAVA} Sitio oficial de Java. Web: \url{http://www.java.com}
\bibitem{FB} Sitio oficial de la API Facebook. \\
Web: \url{http://www.developers.facebook.com}
\bibitem{EC} Sitio oficial de Eclipse. Web: \url{http://www.eclipse.org}
\bibitem{SQL} Sitio oficial de SQLite. Web: \url{http://www.sqlite.org}
\bibitem{XML} Sitio oficial de XML. Web: \url{http://www.w3.org/XML/}
\bibitem{GIT} Sitio oficial de GIT. Web: \url{http://git-scm.com/}
\bibitem{GITH} Sitio oficial de Github. Web: \url{https://www.github.com}
\bibitem{MS} Sitio oficial de Microsoft Project. \\
Web: \url{http://www.microsoft.com/project/}
\end{thebibliography}
\end{document}
